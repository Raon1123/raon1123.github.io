\section{Syntax of First-Order Logic}

\subsection{Alphabets}

\begin{frame}
    \frametitle{Alphabets}
    \begin{definition}
        An \textit{alphabet} $\bA$ is a finite set of \textbf{symbols}. \\
        \bigskip
        \pause
        The finite sequences of symbols from an alphabet $\bA$ are called \textit{strings} or \textit{words}.
        The $\bA^{\star}$ denotes the set of all strings over $\bA$.
    \end{definition}
\end{frame}

\begin{frame}
    \frametitle{Alphabets}
    \begin{Example}
        Suppose that $\bA = \{ a, b, c \}$ is an alphabet. \\
        \bigskip
        \pause
        Then, $\bA^{\star}$ is the set of all strings over $\bA$, i.e.,
        \begin{align*}
            \bA^{\star} = \{ & \square, a, b, c, aa, ab, ac, ba, bb, bc, ca, cb, cc, \\
                             & aaa, aab, aac, aba, abb, abc, aca, acb, acc, \ldots \}
        \end{align*}
        where $\square$ is the \textit{empty string}.
    \end{Example}
\end{frame}

\begin{frame}
    \frametitle{Alphabets}
    \begin{lemma}
        For a nonempty set $M$ the followings are equivalent:
        \begin{enumerate}
            \item $M$ is at most countable. (i.e., $M$ is finite or countably infinite.)
            \item There is an injective map $\beta : M \rightarrow \mathbb{N}$.
            \item There is a surjective map $\alpha : \mathbb{N} \rightarrow M$.
        \end{enumerate}
    \end{lemma}
\end{frame}

\subsection{Alphabets of First-Order Logic}

\begin{frame}
    \frametitle{Alphabets of First-Order Logic}
    
    \begin{definition}
        The alphabet of a first-order language concists of the following symbols:
        \begin{enumerate}
            \item Variables: $v_0, v_1, v_2, \ldots$ \label{def:variables}
            \item $\neg, \wedge, \vee, \rightarrow, \leftrightarrow$ (logical connectives) \label{def:logical_connectives}
            \item $\forall, \exists$ (quantifiers) \label{def:quantifiers}
            \item $\equiv$ (equality) \label{def:equality}
            \item $(, )$ (parentheses) \label{def:parentheses}
            \item For every $n \geq 1$, a (possible empty) set of $n$-ary function symbols \label{def:function_symbols}
            \item For every $n \geq 1$, a (possible empty) set of $n$-ary relation symbols \label{def:relation_symbols}
            \item a (possible empty) set of constant symbols \label{def:constant_symbols}
        \end{enumerate}
    \end{definition}
    Let $\bA$ be the set of symbols \ref{def:variables} to \ref{def:parentheses}.
    Let $S$ be the set of symbols \ref{def:function_symbols}, \ref{def:relation_symbols}, and \ref{def:constant_symbols}. 
    The set $S$ determines the first-order language, for convenience, we denote $\bA_S$ as the alphabet of the first-order language.
\end{frame}

\begin{frame}
    \frametitle{Alphabets of First-Order Logic}

    Question:
    Assume that $S$ is a countable set. Is $\bA_S$ countable?

\end{frame}

\subsection{Terms and Formulas}

\begin{frame}
    \frametitle{Terms and Formulas}
    
    Ok, we define the alphabets of first-order logic. \\
    We already know that not all strings over an alphabet are interpretable. \\
    But, we don't define the meaning (semantics) of first-order logic yet. \\
    \bigskip
    \pause
    We define the \textit{syntax} of first-order logic. It is the grammar of first-order logic. \\

\end{frame}

\begin{frame}
    \frametitle{Terms and Formulas}
    \begin{Definition}\label{def:terms}
        $S$-terms ($T^S$) are precisely those strings in $\bA_S^{\star}$ that can be obtained by applying the following rules:
        \begin{enumerate}
            \item Every variable is an $S$-term.
            \item Every constant symbol is an $S$-term.
            \item If the strings $t_1, \cdots, t_n$ are $S$-terms and $f$ is an $n$-ary function symbol, then $f(t_1, \cdots, t_n)$ is an $S$-term.
        \end{enumerate}
    \end{Definition}
\end{frame}

\begin{frame}
    \frametitle{Terms and Formulas}

    \begin{Definition}\label{def:formulas}
        $S$-formulas ($L^S$) are precisely those strings in $\bA_S^{\star}$ that can be obtained by applying the following rules:
        \begin{enumerate}
            \item If $t_1$ and $t_2$ are $S$-terms, then $t_1 \equiv t_2$ is an $S$-formula.
            \item If $t_1, \cdots t_n$ are $S$-terms and $R$ is an $n$-ary relation symbol, then $R(t_1, \cdots, t_n)$ is an $S$-formula.
            \item If $\phi$ is an $S$-formula, then $\neg \phi$ is an $S$-formula.
            \item If $\phi$ and $\rho$ are $S$-formulas, then $(\phi \wedge \rho)$, $(\phi \vee \rho)$, $(\phi \rightarrow \rho)$, and $(\phi \leftrightarrow \rho)$ are $S$-formulas.
            \item If $\phi$ is an $S$-formula and $x$ is variable, then $\forall x \phi$ and $\exists x \phi$ are $S$-formulas.
        \end{enumerate}
    \end{Definition}

\end{frame}

\begin{frame}
    \frametitle{Terms and Formulas}
    
    We might wonder why we need to define the syntax of first-order logic. \\
    \bigskip
    \pause
    The syntax of first-order logic is construction of strings in first-order logic. \\
    The terms and formulas are the different types of strings in first-order logic. \\
    \bigskip
    \pause
    We borrow the idea from next lecture 'semantics' of first-order logic, the terms are the 'objects' and the formulas are the 'properties' of the objects. \\
    Note that, we don't define the meaning of the terms and formulas yet. Keep in mind it. \\
\end{frame}

\subsection{Induction in the Calculi of Terms and Formulas}

\begin{frame}
    \frametitle{Induction in the Calculi of Terms and Formulas}

    Let $S$ be the set of symbols and $Z \subseteq \bA_S^{\star}$ be the set of strings over the alphabet $\bA_S$. \\
    \bigskip
    \pause
    We want to construct that from symbol set $S$ to the terms $T^S$ \ref{def:terms} and formulas $L^S$ \ref{def:formulas}. \\
    We have the rules to construct the terms and formulas!
\end{frame}

\begin{frame}
    \frametitle{Induction in the Calculi of Terms and Formulas}
    
    We can construct the terms and formulas by inductions. Let's the induction begin! \\
    \bigskip
    \pause
    Assume that $\zeta_1, \cdots , \zeta_n$ all belong to $Z$. Then also $\zeta$ belongs to $Z$ writing as following \\
    \begin{equation}
        \frac{\zeta_1, \cdots , \zeta_n}{\zeta}
    \end{equation}
    \bigskip
    \pause
    We call this rule as \textit{inference rule}. \\
    It allows $n=0$, the first sort of rules in \ref{def:terms} and \ref{def:formulas} is "premise-free" rules. \\
\end{frame}

\begin{frame}
    \frametitle{Induction in the Calculi of Terms and Formulas}

    The calculus $\mathbf{C}$ (rule) of terms $T^S$ as follows: 
    \begin{align}\label{def:calculus_terms}
        \frac{\square}{x} & \\
        \frac{\square}{c}, & c \in S \\
        \frac{t_1, \cdots, t_n}{f(t_1, \cdots, t_n)},& f \in S, f \text{ is $n$-ary function symbol}
    \end{align}

    We can prove that some strings are terms by using the inference rules it is called \textit{induction over calculus $\mathbf{C}$}. \\
\end{frame}

\begin{frame}
    \frametitle{Induction in the Calculi of Terms and Formulas}

    The calculus $\mathbf{C}$ (rule) of formulas $L^S$ as follows:
    \begin{align}\label{def:calculus_formulas}
        \frac{t_1, t_2}{t_1 \equiv t_2} & \\
        \frac{t_1, \cdots, t_n}{R(t_1, \cdots, t_n)}, R \in S&, \quad R \text{ is $n$-ary relation symbol} \\
        \frac{\phi}{\neg \phi}, & \\
        \frac{\phi, \rho}{(\phi \star \rho)}, & \quad \text{where} \star = \vee, \wedge, \rightarrow, \leftrightarrow \\
        \frac{\phi}{\forall x \phi}, & \\
        \frac{\phi}{\exists x \phi}, &
    \end{align}
\end{frame}

\begin{frame}
    \frametitle{Induction in the Calculi of Terms and Formulas}

    \begin{Definition}\label{def:variables}
        The function $\textrm{var}$, which associates with each $S$-terms the set of variables ocurring in it:
        \begin{align*}
            \var({x}) &:= \{x\} \\
            \var({c}) &:= \emptyset \\
            \var({f(t_1, \cdots, t_n)}) &:= \var{t_1} \cup \cdots \cup \var{t_n}
        \end{align*}
    \end{Definition}
    
\end{frame}

\begin{frame}
    \frametitle{Induction in the Calculi of Terms and Formulas}

    \begin{definition}\label{def:subformulas}
        The function $\SF$, which assigns to each formula the set of its subformulas as following:
        \begin{align*}
            \SF(t_1 \equiv t_2) &:= \{t_1 \equiv t_2\} \\
            \SF(R(t_1, \cdots, t_n)) &:= \{R(t_1, \cdots, t_n)\} \\
            \SF(\neg \phi) &:= \{\neg \phi\} \cup \SF(\phi) \\
            \SF((\phi \star \rho)) &:= \{(\phi \star \rho)\} \cup \SF(\phi) \cup \SF(\rho) & \quad \text{where} \star = \vee, \wedge, \rightarrow, \leftrightarrow \\
            \SF(\forall x \phi) &:= \{\forall x \phi\} \cup \SF(\phi) \\
            \SF(\exists x \phi) &:= \{\exists x \phi\} \cup \SF(\phi)
        \end{align*}
    \end{definition}

\end{frame}

\begin{frame}
    \frametitle{Free variables}

    \begin{definition}\label{def:free}
        The function $\free$, which assigns to each formula the set of its free variables as following:
        \begin{align*}
            \free(t_1 \equiv t_2) &:= \var(t_1) \cup \var(t_2) \\
            \free(R(t_1, \cdots, t_n)) &:= \var(t_1) \cup \cdots \cup \var(t_n) \\
            \free(\neg \phi) &:= \free(\phi) \\
            \free((\phi \star \rho)) &:= \free(\phi) \cup \free(\rho) & \quad \text{where} \star = \vee, \wedge, \rightarrow, \leftrightarrow \\
            \free(\forall x \phi) &:= \free(\phi) \setminus \{x\} \\
            \free(\exists x \phi) &:= \free(\phi) \setminus \{x\}
        \end{align*}
    \end{definition}
\end{frame}

\begin{frame}
    \frametitle{Summary}

    For summary of this lecture, we have the following definitions:
    \begin{itemize}
        \item \textbf{Terms} $T^S$ \ref{def:terms}
        \item \textbf{Formulas} $L^S$ \ref{def:formulas}
        \item \textbf{Induction over calculus $\mathbf{C}$} \ref{def:calculus_terms} and \ref{def:calculus_formulas}
        \item \textbf{Subformulas} \ref{def:subformulas}
        \item \textbf{Free variables} \ref{def:free}
    \end{itemize}
    
\end{frame}