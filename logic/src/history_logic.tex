\section{History of Logic}


\begin{frame}
    \frametitle{History of Logic: Acient Logic}

    Euclid's \textit{Elements} (300 BC) is the earliest known uses of logic and proofs.
    In \textit{Elements}, Euclid uses a deductive axiomatic method to study geometry.
    Euclid's method of proving mathematical statements is called \textit{axiomatic method}.
    In axiomatic method, we start with a set of axioms and use rules of inference to derive theorems.
    In \textit{Elements}, Euclid uses five postulates and five common notions as axioms.
    Euclid's axioms are not formalized, but they are the first known examples of axioms.

    Aristotle (384-322 BC) is the first known logician who studied logic systematically.
    Aristotle's logic is called \textit{term logic} because it is concerned with terms rather than propositions.

\end{frame}

\begin{frame}
    \frametitle{History of Logic: Boolean Logic}

    In the 19th century, George Boole (1815-1864) and Augustus De Morgan (1806-1871) developed a new kind of logic called \textit{Boolean logic}.
    Before Boole and De Morgan, we prove statements using predicate logic.
    But, Boole and De Morgan developed a new kind of logic that is based on algebra.
    In Boolean logic, we use algebraic operations such as $\land$, $\lor$, $\lnot$, and $\Rightarrow$ to prove statements.
    
\end{frame}

\begin{frame}
    \frametitle{History of Logic: Hilbert's Program}

    In the 20th century, David Hilbert (1862-1943) proposed a program to formalize all of mathematics.
    Hilbert's program is based on the idea that all of mathematics can be formalized in first-order logic.
    Hilbert's program is called \textit{formalism}.

\end{frame}