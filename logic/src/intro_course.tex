\section{Introduction to Course}

\begin{frame}
    \frametitle{Why we study Mathematical Logic?}

    In this course, we will study mathematical logic and computability theory.
    What is the \textit{logic}? Why we need it for us?

    Logic is the study of the principles of correct reasoning.
    Logic consider formal or informal language and prove that with induction or model-theoretic method.
    Among logic, mathematical logic is to find reasoning in mathematics (with mathematical reasoning).
    We will study and explore the \textit{first-order logic} for expressing mathematics formally.

\end{frame}

\begin{frame}
    \frametitle{Why we study Computability?}
    
    Let's think about another simple question: What is computable?
    We can say about that "We can compute 1+1=2, 2+2=4, 3+3=6, ...". But, how can we say about that?
    
    How can we construct a machine that can compute 1+1=2, 2+2=4, 3+3=6, ...?
    Can we compute all of the functions that we can imagine?
    We already know about the answer of this question is '\textbf{no}' by the Halting Problem.
    But we need to consider foundation of this answer, computation model.

\end{frame}

\begin{frame}
    \frametitle{Outline of Course}

    In this course, we will study following topics in mathematical logic:
    \begin{itemize}
        \item Syntax of First-Order Logic
        \item Semantics of First-Order Logic
        \item Sequent Calculus
        \item Completeness Theorem
        \item The L\"owenheim-Skolem Theorem and Compactness Theorem
        %\item Formal Verification (Hoare Logic, Herbrand's Theorem, SAT Solvers)
    \end{itemize}

\end{frame}

\begin{frame}
    \frametitle{Outline of Course}
    
    In this course, we will study following topics in computability theory:
    \begin{itemize}
        \item Introduction to Computable Functions by Unlimited Register Machine
        \item Build on Computable Functions
        \item Turing Machines and Church-Turing Thesis
        \item Numbering of Computable Functions
        \item s-m-n Theorem, Universal Programs
        \item Deciability and Undecidability (Halting Problem, Rice's Theorem)
    \end{itemize}
\end{frame}

\begin{frame}
    \frametitle{Course in Catalog}
    
    Before this course, you have taken following courses:
    \begin{itemize}
        \item Discrete Mathematics
        \item Programming Languages
        \item Introduction to Algorithms
    \end{itemize}

    After this course, you will discuss following topics:
    \begin{itemize}
        \item Type, Proof, Model Theory
        \item Set Theory
        \item Computability Theory
        \item (further) Computational Complexity Theory
    \end{itemize}
\end{frame}

\begin{frame}
    \frametitle{Textbook}
    
    In this course, in mathematical logic, we will use following textbook:
    \begin{itemize}
        \item \textit{Mathematical Logic}, by H.-D. Ebbinghaus, J. Flum, and W. Thomas
    \end{itemize}

    In this course, in computability theory, we will use following textbook:
    \begin{itemize}
        \item \textit{Computability: An Introduction to Recursive Function Theory}, by N. J. Cutland
    \end{itemize}
\end{frame}

\begin{frame}
    \frametitle{Textbook}

    In this course, I recommend following textbook:
    \begin{itemize}
        \item \textit{Introduction to Mathematical Logic}, by E. Mendelson
        \item \textit{Mathematical Logic}, by J. Shoenfield
        \item \textit{Computability and Logic}, by G. S. Boolos, J. P. Burgess, and R. C. Jeffrey
        \item \textit{Computability and Unsolvability}, by M. Davis
    \end{itemize}
    
\end{frame}


