\section{L\"owenheim-Skolem Theorem}

\begin{frame}
    \frametitle{L\"owenheim-Skolem Theorem}

    \begin{block}{L\"owenheim-Skolem Theorem}
        Every at most countable and satisfiable set of formulas is satisfiable over a domain which is at most countable. \\
        i.e., it has a model whose domain is at most countable.
    \end{block}
    
\end{frame}

\begin{frame}
    \frametitle{Proof Sketch}
    
    Let $\Phi$ be an at most countable set of $S$-sentences which is satisfiable and hence consistent. \\
    $S$-formula contains only finitely many $S$-symbols, there are at most countably many S-symbols in $\Phi$. \\
    We may without loss of generality assume that $S$ itself is at most countable. \\
    Since $\Phi$ is satisfiable, $\Phi$ is consistent. \\
    There is an interpretation which satisfies $\Phi$ and whose domain $A$ consists of equavalence classes of terms $\overline{t}$ of $S$, where $t$ ranges over $T^S$. \\
    Because $T^S$ is at most countable, $A$ is at most countable. \\
\end{frame}

\begin{frame}
    \frametitle{Downward L\"owenheim-Skolem Theorem}
    
    \begin{Theorem}
        If $\Phi \subseteq L^S$ is a set of $S$-sentences which is satisfiable over a domain $A$, 
        then it is satisfiable over a domain of cardinality not greater than the cardinality of $L^S$.
    \end{Theorem}

\end{frame}

\begin{frame}
    \frametitle{Meaning of the Downward L\"owenheim-Skolem Theorem}
    
    In our language, include this sentence, we use at most countable many symbols. \\
    For example, when we represent sentence domain on the real number, we use at most countable many symbols. \\
    The set of real number sentence is countable, then it is satisfiable over at most countable domain. \\
    But, we already know about the real number set is not countable! \\
    Is it constradiction?
\end{frame}